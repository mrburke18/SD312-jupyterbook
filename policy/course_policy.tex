%% LyX 2.2.1 created this file.  For more info, see http://www.lyx.org/.
%% Do not edit unless you really know what you are doing.
\documentclass[11pt,english]{article}
\usepackage{mathptmx}
\renewcommand{\familydefault}{\rmdefault}
\usepackage[T1]{fontenc}
\usepackage[latin9]{inputenc}
\usepackage{geometry}
\geometry{verbose,tmargin=1in,bmargin=1in,lmargin=1in,rmargin=1in}
\setlength{\parindent}{0bp}
\PassOptionsToPackage{normalem}{ulem}
\usepackage{ulem}

\makeatletter

%%%%%%%%%%%%%%%%%%%%%%%%%%%%%% LyX specific LaTeX commands.
%% Because html converters don't know tabularnewline
\providecommand{\tabularnewline}{\\}

\makeatother

\usepackage{babel}
\begin{document}
\begin{center}
  SD312: Machine Learning
\par\end{center}

\begin{center}
Course Policy, Spring AY24
\par\end{center}

\setlength{\tabcolsep}{0pt}
\begin{tabular}{ll}
  \uline{Instructors}:& Prof. Gavin Taylor (Coordinator), HP455, x3-6816, taylor@usna.edu\\
  & CDR Ed Jatho, HP461, x3-6812, jatho@usna.edu\\
\end{tabular}\\

\uline{Course Description}: Machine Learning is the study of mathematically
making autonomous conclusions about new data given insight from
previously-seen data.  This course is intended to take the student beyond linear regression and classification models as taught in SM317. Subjects include learning theory, unsupervised learning, recommendation systems, reinforcement learning, and neural networks.\\

\uline{Credits}: 2-2-3\\

\uline{Pre-requisites}: SM317 and (SD311 or SI312)\\

\uline{Student Outcomes}:\\

Graduates of the program will have an ability to:\\ 
\quad{}1.  Analyze a complex computing problem and to apply principles of computing and other relevant disciplines to identify solutions.\\
\quad{}2.  Design, implement, and evaluate a computing-based solution to meet a given set of computing requirements in the context of the program's discipline.\\
\quad{}3.  Communicate effectively in a variety of professional contexts. \\
\quad{}4.  Recognize professional responsibilities and make informed judgments in computing practice based on legal and ethical principles.\\
\quad{}5.  Function effectively as a member or leader of a team engaged in activities appropriate to the program's discipline.\\
\\
\quad{}DS-6.  Apply theory, techniques and tools throughout the data analysis lifecycle and employ the resulting knowledge to satisfy stakeholders' needs.\\

In this course, we will focus on assessing (3) in a written context and (DS-6).\\

\uline{Textbook(s)}: Online books will be used, and will be linked to on the
resources page on the course website.\\

\uline{Syllabus}:
\begin{itemize}
  \item Unsupervised Learning (1 week)
  \item Recommendation Systems (1 week)
  \item Learning Theory (1 week)
  \item Optimization (1 week)
  \item SVMs (1 week)
  \item Neural Networks/Convolutional Networks (4 weeks)
  \item Reinforcement Learning (4 weeks)
  \item Ethics (1 week)
\end{itemize}

\uline{Extra Instruction}: Extra instruction (EI) is strongly encouraged
and should be scheduled by email with the instructor. EI is not a
substitute lecture; students should come prepared with specific questions
or problems.\\

\uline{Grading Policy and Collaboration}: The guidance in the Honor Concept of
the Brigade of Midshipmen and the Computer Science Department Honor Policy must
be followed at all times. See 
\\\texttt{www.usna.edu/CS/resources/honor.php}. Specific instructions for this
course:
\begin{itemize}
\item Assignments: There will be many assignments.  You may use any source for help, and discuss them with anybody,
  but \textbf{all submitted work must be yours}, and all help must be
    documented.  If you cannot explain each line of the work, it is clearly not your work!

    Using \textbf{free} generative AI tools like ChatGPT will be allowed in certain circumstances, though you are responsible for your work being correct in content and nuance.  If Generative AI is not explicitly allowed, you should assume it is disallowed.  Every assignment must begin with a Generative AI Citation, explaining what content was created by generative AI, and what specific tools were used to produce it.
\item Exams: There will be two midterms and a comprehensive final.  Should a
  make-up exam be needed, inform the instructor at least one week in advance.
\end{itemize}

\uline{Classroom Conduct}: The section leader will record attendance and bring
the class to attention at the beginning and end of each class. If the
instructor is late more than 5 minutes, the section leader will keep the class
in place and report to the Computer Science department office. If the
instructor is absent, the section leader will direct the class. Drinks are
permitted, but they must be in reclosable containers. Food, alcohol, smoking,
smokeless tobacco products, and electronic cigarettes are all prohibited. Cell
phones must be silent during class. All discussions will be civil, and both
faculty and midshipmen will be treated with dignity and respect at all
times.\\
\\
\uline{Late Policy}: Nothing will be accepted late without a good reason.\\
\\
\uline{Grading}: Student performance will be translated to the following
letter grades: A, A-, B+, B, B-, C+, C, C-, D+, D, F.  The breakdown of the
final course grade will be:
\begin{itemize}
\item \textbf{25\% Final Exam} - the final exam will be cumulative.
\item \textbf{25\% Mid-term Exams} - The mid-term is written.  There may be
    a component requiring a computer.
\item \textbf{50\% Programming Assignments} - Detailed instructions for the
electronic submission will accompany each project.  Our lab time will
be dedicated to these assignments, but they are not intended to be completely
finished in class.  You should expect these to take some time.
\end{itemize}

\uline{Submitted:} Professor Gavin Taylor

\end{document}
